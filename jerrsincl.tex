Questo capitolo focalizza l'attenzione sul lavoro che ha gettato le fondamenta per lo sviluppo del e su cui si basa questa tesi e tanti altri lavori. L'articolo di cui si parla è ``\textit{Polynomial-time approximation algorithms for the Ising model}'', scritto da Mark Jerrum ed Alistair Sinclair nel 1993 \cite{jerrum1993polynomial}.\\
In breve, \cite{jerrum1993polynomial} presenta un algoritmo randomizzato che calcola la partition function di un sistema di Ising ferromagnetico qualunque, con grado di accuratezza arbitrario. Il tempo di esecuzione di tale algoritmo è polinomiale nella taglia del sistema (i.e. il numero di siti) e variabile in base ad un parametro che controlla l'accuratezza del risultato.
L'algoritmo si basa su una simulazione Monte Carlo di una catena di Markov ergodica opportunamente definita. Si può già anticipare che gli stati della catena non sono, come accade di solito, configurazioni di spin di Ising, ma spanning subgraph del grafo di interazione del sistema. Le performance dell'algoritmo sono garantite da prove rigorose e poggiano sulla proprietà di \textit{rapidly mixing} della catena di Markov: converge alla sua distribuzione d'equilibrio in un numero \textit{polinomiale} di passi.\\
Si prosegue con l'analisi dettagliata dell'algoritmo proposto da Jerrum e Sinclair, descrivendo prima il modo con cui si trasforma il modello di Ising in un nuovo dominio, in cui le configurazioni sono spanning subgraph del grafo di interazione; prosegue con la costruzione di un \textit{fully polynomial randomised approximation scheme (fpras)} per il calcolo della partition function ed infine analizzando la catena di Markov definita sulle nuove configurazioni.\\
Nei capitoli successivi, l'algoritmo proposto verrà identificato come ``\textit{algoritmo naive}''.
\section{Spins world e Subgraphs world}
L'obiettivo è costruire un algoritmo per il seguente problema:
\textbf{Istanza}: una matrice reale simmetrica $(V_{ij} : i,j \in [n])$ delle energie di interazione, un numero reale B che rappresenta il campo esterno ed un numero reale positivo $\beta$.
\textbf{Output}: la partition function
\begin{equation}
	Z = Z(V_{ij}, B, \beta) = \sum_{\sigma}{exp(-\beta H(\sigma))},
	\label{partf}
\end{equation}
Dove l'Hamiltoniana $H(\sigma)$ è data da
\begin{equation}
	H(\sigma) = - \sum_{\{i,j\}\in E}{V_{ij}\sigma_i\sigma_j} - B\sum_{k\in [n]}{\sigma_k}
	\label{hamilt}
\end{equation}
ed E è l'insieme di coppie non ordinate ${i, j}$ con $V_{ij} \ne 0$.\\
Tale algoritmo tratta il caso \textit{ferromagnetico} del modello di Ising, che è caratterizzato da energie di interazione $V_{ij}$ non negative. È importante tenere a mente che piuttosto di calcolare \textit{esattamente} la partition function, si preferisce approssimarla.\\
Una strategia, rivelatasi molto utile per problemi di questo tipo, prevede la simulazione di una catena di Markov opportuna. Un'applicazione diretta di questa strategia alla partition function di Ising procederebbe come segue: si considerino le configurazioni del sistema di Ising, cioè i $2^n$ possibili vettori di spin $\sigma \in {-1, +1}^n$, come gli stati della catena di Markov. Si scelgano poi le probabilità di transizione tra gli stati così da rendere la catena \textit{ergodica} e quindi, nella distribuzione stazionaria, la probabilità di essere nello stato $\sigma$ è $Z^{-1}exp(-\beta H(\sigma))$. Un modo ragionevole per raggiungere ciò, spesso utilizzato, è permettere che le transizioni occorrano tra configurazioni di sping che differiscano in una sola componente, e scegliere le probabilità di transizione in accordo alla regola di Metropolis \cite{gelatt1983optimization}. Se la catena di Markov risultante è \textit{rapidly mixing}, cioè se converge rapidamente alla distribuzione stazionaria indipendentemente dalla scelta dello stato iniziale, allora può essere usata efficacemente per campionare le configurazioni $\sigma$ da una distribuzione che è vicina alla distribuzione stazionaria. Raccogliendo sufficienti campioni di configurazioni, usando differenti valori di B e $\beta$, dovrebbe essere possibile stimare la partition function $Z$ con buona accuratezza.\\
Purtroppo, la catena di Markov così descritta (lo \textit{spin-world process}) non è rapidly mixing. È ben noto che i sistemi ferromagnetici di Ising esibiscono tipicamente una transizione di fase ad un certo valore del parametro $\beta$, per valori al di sopra del punto critico, il sistema si stabilizza in uno stato in cui vi è una preponderanza di spin di uno o l'altro segno. transizioni tra gli stati con maggioranza di +1 e stati con maggioranza di -1 occorrono raramente, semplicemente perché la distribuzione stazionaria assegna un peso totale di basso valore alle configurazioni con spin bilanciati.\\
Il problema causato dall'assenza di \textit{rapid mixing} nello \textit{spin-world process} può essere aggirato simulando una catena di Markov differente: il \textit{subgraphs-world process}. Sebbene le due catene di Markov siano strutturalmente differenti, ed inoltre, il \textit{subgraphs-world process} non abbia alcun significato fisico, quest'ultimo ha una forte connessione con la \textit{partition function} del modello di Ising e, cosa fondamentale nell'applicazione corrente, è \textit{rapidly mixing}.\\
Tale processo sarà descritto in dettaglio nella sezione \ref{sec:swp}.\\\\
Un sottografo si dice \textit{spanning} se include tutti i vertici del grafo ``genitore'' (in generale gli \textit{spanning subgraph} non sono connessi). Le configurazioni del \textit{subgraph-world} sono \textit{spanning subgraph} del grafo di interazione $([n], E)$. Per semplificare la notazione, siano
\begin{equation}
	\lambda_{ij} = tanh \beta V_{ij}
	\label{lamij}
\end{equation}
\begin{equation}
	\mu = tanh B \beta
	\label{mu}
\end{equation}
Ad ogni configurazione $X \subseteq E$ è assegnato un \textit{peso}, in accordo alla formula
\begin{equation}
	w(X) = \mu^{|odd(x)|}\prod_{\{i,j\} \in X}{\lambda_{ij}},
	\label{weightfunc}
\end{equation}
in cui la notazione odd(X) sta ad indicare l'insieme di tutti i vertici che hanno grado dispari nel grafo $X$.\\
La partition function per il \textit{subgraphs-world} è
\begin{equation}
	Z^\prime = \sum_{X \subseteq E}{w(X)}.
	\label{zprime}
\end{equation}
La formula \ref{zprime} è conosciuta come ``\textit{the high temperature expansion}''.\\
Una relazione interessante è quella tra le partition function $Z$ e $Z^\prime$: esse sono correlate in maniera semplice. Si definisca
\begin{equation}
	A = (2cosh\beta B)^{n} \prod_{\{i,j\}\in E}{cosh\beta V_{i,j}},
	\label{funca}
\end{equation}
si noti che $A$ è una funzione che può essere facilmente calcolata, poiché è composta dai parametri che specificano il sistema di Ising.\\
Il seguente risultato classico \cite{newell1953theory} lega le due partition function:
\begin{thm}
	$Z = AZ^\prime$.
	\label{thm:zaz}
\end{thm}
Tale teorema ha portato gli autori di questo lavoro a considerare un sistema della meccanica statistica le cui configurazioni sono \textit{spanning subgraph} di $([n], E)$. In seguito sarà definita una catena di Markov i cui stati sono queste configurazioni, e la cui distribuzione stazionaria assegna probabilità $\pi(X) = w(X)/Z^\prime$ alla configurazione $X$. questo processo, analizzato in dettaglio nella sezione \ref{sec:swp}, sarà dimostrato essere \textit{rapidly mixing} ed, inoltre, fornirà un mezzo efficiente per campionare le configurazioni con probabilità approssimativamente proporzionali ai loro pesi.\\
Poiché $Z^\prime$ è una somma pesata delle configurazioni, ci si aspetta che tale procedura dia informazioni utili su $Z^\prime$ stessa e, di conseguenza, sulla partition function originale $Z$.\\
Per la dimostrazione di tale teorema, riferirsi a \cite{jerrum1993polynomial} (pag. 1091).
\section{Stima della Partition Function}
Si analizza ora, e descrive, un \textit{efficiente} algoritmo di approssimazione per il calcolo della partition function $Z$ di un sistema di Ising ferromagnetico (definizione di ``algoritmo di approssimazione efficiente'' al paragrafo \ref{sec:approxalgo}).\\
Gli autori dell'articolo assumono di trattare con un modello computazionale in cui l'aritmetica è effettuata con un'accuratezza perfetta ed in cui le operazioni aritmetiche e le funzioni standard (i.e. l'esponenziazione) abbiano costo unitario. Inoltre, la taglia dell'istanza del problema è data da $n$, cioè il numero di siti.\\
Dato un sistema di Ising ferromagnetico $\langle\lambda_{ij}, \mu\rangle$  on $\lambda_{ij}$ e $\mu$ definiti nella sezione precedente in \ref{lamij} e \ref{mu}, sia $\Omega$ l'insieme di \textit{subgraphs-world configuration} e sia $\pi$ la distribuzione di probabilità su $\Omega$: $\pi(X) = w(X)/\sum_{X^{\prime}}{w(X^{\prime})} = w(X)/Z^{\prime}$, dove $w$ è la funzione peso definita in \ref{weightfunc}.
È importante notare che, poiché il sistema è ferromagnetico, $w(X) \geq 0$ per ogni $X \in \Omega$, pertanto $\pi$ è una distribuzione di probabilità.
\paragraph{Generatore.} Un \textit{generatore} per le \textit{subgraphs-world configuration} è un algoritmo probabilistico che prende in input un sistema di Ising ferromagnetico nella forma $\langle\lambda_{ij}, \mu\rangle$, più una tolleranza reale positiva $\delta$, e restituisce un elemento di $\Omega$ preso da una distribuzione $p$ che soddisfa la seguente \textit{variation distance}
\begin{equation}
	\label{vardist}
	\|p-\pi\| \leq \delta.
\end{equation}
Risulta possibile costruire un generatore efficiente per le \textit{subgraphs-world configuration}, come enunciato nel seguente teorema.
\begin{thm}
	Esiste un generatore per le subgraphs-world configuration che, su input $\langle \lambda_{ij}, \mu\rangle$ e $\delta$, ha un tempo d'esecuzione limitato da un polinomio in $n$, $\mu^{-1}$ e $log\delta^{-1}$. Nello specifico, il tempo di esecuzione del generatore è $O(m^2\mu^{-8}(log\delta^{-1} + m))$, in cui $m = |E|$ è il numero di interazioni non nulle.
	\label{thm:gen}
\end{thm}
È importante notare che la presenza di $\mu^{-1}$ nel bound sul tempo implica che il generatore sia inefficiente per sistemi con un campo esterno molto basso. Come detto in precedenza, la costruzione di un generatore con tali proprietà basato sulla simulazione di una catena di Markov opportunamente definita, è descritta e giustificata nella sezione \ref{sec:swp}. Per il momento gli autori dell'articolo assumono il Teorema \ref{thm:gen} e mostrano come i campioni prodotti dal generatore possano essere usati per ottenere un efficiente algoritmo di approssimazione per la partition function Z.\\
Jerrum e Sinclair partono da una considerazione: suppongono di voler stimare il valore di una quantità fisica associata ad un sistema di Ising ferromagnetico. Il primo passo è esprimere tale quantità come \textit{valore atteso} di una \textit{variabile casuale} opportunamente definita sulle configurazioni del \textit{subgraphs world}. Dopodiché assumono di poter stimare tale quantità campionando delle configurazioni in maniera casuale con l'aiuto del generatore del Teorema \ref{thm:gen}, e calcolandone infine la \textit{media campionaria}.
Precisamente, sia $f$ una funzione non negativa a valori reali definita sull'insieme $\Omega$ di configurazioni del \textit{subgraphs world} di un sistema di Ising ferromagnetico. Considerando $\Omega$ come uno spazio campionario con distribuzione di probabilità $\pi(X) = w(X)/Z^{\prime}$, la funzione $f$ diventa una variabile casuale, con valore atteso
\begin{equation}
	E(f) = \frac{1}{Z^{\prime}}\sum_{X\in\Omega}{w(X)f(X)}.
	\label{e_f}
\end{equation}
Partendo dal generatore del Teorema \ref{thm:gen} è semplice ottenere una stima di $E(f)$: si costruisce un campione indipendente ${X_i}$ di configurazioni di taglia $s$, e si calcola la media campionaria $s^{-1}\sum_i{f(X_i)}$. Facendo sì che la taglia $s$ sia abbastanza grande, si può raggiungere un qualsiasi grado di accuratezza desiderato con una ragionevole affidabilità. Inoltre, si può ridurre drasticamente la probabilità che la stima cada al di fuori del range accettabile di accuratezza ripetendo l'intero processo $t$ volte e calcolando la \textit{mediana} dei $t$ risultati. L'efficienza di tale esperimento, quindi, dipende da quanto sono grandi i valori di $s$ e $t$; ciò, a sua volta, dipende dalla \textit{varianza} della variabile casuale $f$ o, più precisamente, dal rapporto $max(f)/E(f)$, dove $max(f)$ denota il valore massimo che $f$ assume su $\Omega$.
Il seguente Lemma quantifica i risultati ottenuti:
\begin{lem}
	Sia f una variabile casuale a valori reali e non negativi definita sull'insieme $\Omega$ di subgraphs-world configuration di un sistema di Ising ferromagnetico, e siano $\xi$, $\eta$ numeri reali tali che $0 < \xi \leq 1$ e $0 < \eta \leq 1/2$. Allora vi è un esperimento nella forma descritta in precedenza che utilizza un totale di $504\xi^{-2}\lceil lg\eta^{-1} \rceil max(f)/E(f)$ campioni dal generatore, ognuno con input $\langle\lambda_{ij}, \mu \rangle$ e tolleranza $\delta = \xi E(f)/8max(f)$, e produce un output $Y$ che soddisfa: $Pr(Y \text{ approssima } E(f) \text{ in un range } 1+\xi) \geq 1 - \eta$.
	\label{lem:gensteps}
\end{lem}
Data l'importanza ed il ruolo cruciale che tale Lemma ha avuto nello sviluppo di questo lavoro di tesi, la sua dimostrazione è descritta per esteso e nel dettaglio nel paragrafo \ref{ssec:lemmaproof}.\\
Il Lemma \ref{lem:gensteps} afferma chiaramente che, qualora si impiegasse tale tecnica, è necessario assicurarsi che il rapporto $max(f)/E(f)$ non sia troppo grande per la variabile casuale $f$ in considerazione. In particolare per questo caso, il criterio di efficienza adottato prevede che il rapporto sia limitato da una funzione polinomiale in $n$, la taglia del sistema.\\
Si analizzi ora com'è possibile applicare tale tecnica per calcolare la partition function $Z(V_{ij}, V, \beta)$ richiamando anche il Teorema \ref{thm:zaz} della sezione precedente (ricordando che la funzione $A$ può essere calcolata direttamente). In accordo a tale teorema, ci si concentra principalmente sul calcolo di $Z^{\prime}$; il primo passo è quello di scrivere $Z^{\prime}$ esplicitamente in funzione di $\mu$ come segue:
\begin{equation}
	Z^{\prime} \equiv Z^{\prime} = \sum_{X \subseteq E}{\mu^{|odd(X)|}} \prod_{\{i,j\}\in X}{\lambda_{ij} = \sum_{k=0}^{\lfloor n/2 \rfloor}}{c_k\mu^{2k}}.
\end{equation}
È importante notare che solo le potenze pari di $\mu$ devono essere incluse nella somma, poiché il numero di vertici di grado dispari nel sottografo $X$ è necessariamente pari (si veda il Teorema di Eulero). Pertanto, si considera $Z^{\prime}$ come un polinomio in $\mu^2$ con coefficienti
\begin{equation}
	c_k = \sum_{X:|odd(X)|=2k}{\prod_{\{i,j\}\in X}}{\lambda_{ij}}.
\end{equation}
\subsection{Prova del Lemma 4.2.1}\label{ssec:lemmaproof}
\section{Analisi del subgraphs-world process} \label{sec:swp}