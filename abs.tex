\documentclass[11pt,oneside]{book}

% per stampare fronteretro
%\documentclass[11pt,twoside]{book}

%\captionheaderfont{\sl\bfseries}
%\captionbodyfont{\sl}

%%\renewcommand{\tableshortname}{Table}
%%\renewcommand{\figureshortname{Figure}
%%\chapapp{\chaptername}
\usepackage[italian,english]{babel}
\usepackage[latin1]{inputenc}
\usepackage{epsfig}
\usepackage{psfig}
\usepackage{subfig}
\usepackage{cite}
\usepackage{amsmath}
\usepackage{latexsym}
%\usepackage[italian]{minitoc}
\usepackage{fancybox}

\usepackage{psfig}
\usepackage{verbatim}
\usepackage{url}
\usepackage{graphicx}
\usepackage{color}
%\usepackage{concmath}
\usepackage{moreverb}
%HYPERREF per uso con dvi
%\usepackage[final,backref,breaklinks,pagebackref,colorlinks]{hyperref}
%HYPERREF per uso con con pdf (con bookmark)
\usepackage[final,bookmarks,backref,breaklinks,dvips,colorlinks]{hyperref}
%mypackages
\usepackage{listings}

\newlength{\realtextwidth}

%%%%% marginpar macros and new realtextwidth
\newcommand{\margintext}[1]
{\mbox{}\marginpar{\sf\tiny\bf\hspace{0pt}#1}}
\marginparwidth 60pt
\setlength{\realtextwidth}{380pt}
%\setlength{\realtextwidth}{\textwidth}
%\addtolength{\realtextwidth}{\marginparwidth}
%\addtolength{\realtextwidth}{\marginparsep}




%%%%%%%%%%%%%%%%%%%%%%%%
%--------------------------------------------------
% MACROs
%--------------------------------------------------
\newcommand{\commento}[1]{\begin{quote}{\small\it #1}\end{quote}}
\newenvironment{javacode}{\begin{changemargin}{30pt}{0pt}\ \nopagebreak[4]\newline\noindent\small
%\smallskip\hrulefill\newline\noindent%linea da eliminare 
}
{\noindent\nopagebreak[4]\normalsize
\end{changemargin}\noindent
%\hrulefill\smallskip%linea  da eliminare
}
\newcommand{\codice}[1]{{\tt #1}}


\newcommand{\nota}[1]{\begin{center}\fbox{{\footnotesize\tt #1}}\end{center}}


\newenvironment{boxedpar}{\begin{center}
\begin{tabular}{|p{\figtextwidth}|} \hline}{\\ \hline
\end{tabular}\end{center}}

%per la visualizzazione di interazioni a video (compilazioni etc.)
\newenvironment{monitor}{\begin{boxedpar}\begin{verbatim}}
{\end{verbatim}\end{boxedpar}}


% Environment changemargine: Esempio: \begin{changemargin}{0pt}{-60pt}
\newenvironment{changemargin}[2]{%
 \begin{list}{}{%
  \setlength{\topsep}{0pt}%
  \setlength{\leftmargin}{#1}%
  \setlength{\rightmargin}{#2}%
  \setlength{\listparindent}{\parindent}%
  \setlength{\itemindent}{\parindent}%
  \setlength{\parsep}{\parskip}%
 }%
\item[]}{\end{list}}

\newcommand{\rosso}{\centerline{\color[rgb]{1.00,0.00,0.00}\fbox{Rosso}}}
\newcommand{\verde}{\centerline{\color[rgb]{0.00,1.00,0.00}\fbox{Verde}}}
%-------------------------------- end MACROs



%%%%% Larghezza e lunghezza massima delle figure
\newlength{\figtextwidth}
\setlength{\figtextwidth}{\textwidth}
\addtolength{\figtextwidth}{-12pt}
\newlength{\smallfigtextwidth}
\setlength{\smallfigtextwidth}{\textwidth}
\addtolength{\smallfigtextwidth}{-70pt}
\newlength{\figmaxheight}
\setlength{\figmaxheight}{\textheight}
\addtolength{\figmaxheight}{-2truecm}

% per il package verbatimtab:la dimensione del tab
\renewcommand\verbatimtabsize{4\relax}





\begin{document}
\selectlanguage{italian}

\begin{titlepage}
\begin{center}
\begin{center}
\includegraphics[scale=0.28, natwidth=793, natheight=1123]{Figure/logounisa.jpg}
\end{center}
{\Large Universit� degli Studi di Salerno}\\[0.2truecm]
{\large Facolt� di Scienze Matematiche Fisiche e Naturali}\\
\hrulefill
\vfill
{\large Tesi di Laurea di I livello in }\\[0.2truecm]
{\Large Informatica}\\
\vfill
{\Huge Abstract}\\
\vfill
{\Huge Progettazione assistita di simulazioni agent-based: l'architettura di Agent Modeling Platform}\\
\vfill\vfill


\ \ \ \ \ \ \ {\bf Relatore} \hfill {\bf Candidato}\ \ \\
Prof. Vittorio Scarano \hfill Francesco Farina\\
\hfill Matricola 0512100694

\vfill
\hrulefill 

Anno Accademico 2012-2013

\end{center}
\end{titlepage}

\chapter*{Abstract}
Lo studio dei sistemi multiagente (ABM) ha acquisito grande attenzione nell'ultimo decennio ed � diventato uno strumento sempre pi� diffuso in molti ambiti di ricerca come biologia, economia, ecologia, scienze sociali.\\
I sistemi multiagente nascono dalla necessit� di rappresentare determinate realt� di interesse, mediante la simulazione basata su agenti ovvero software capace di modellare i comportamenti di un sistema, ed � utilizzata per analizzare sistemi complessi. Grazie alla continua evoluzione del settore informatico, si possiedono le capacit� di modellare e prevedere approssimativamente l'evoluzione di sistemi molto complessi su larga scala.\\
Tipicamente la realizzazione di modelli di simulazioni, rappresentanti la realt� d'interesse, � relegata esclusivamente a sviluppatori software. 
Su questa considerazione nasce l'\textbf{Agent Modeling Platform}, progetto Eclipse open source, che si pone come obiettivo offrire la possibilit� di costruire modelli di simulazione, anche a chi non possiede conoscenze nel campo della programmazione, attraverso l'utilizzo di strumenti grafici intuitivi.\\
AMP fornisce il supporto per diversi sistemi multiagente e consente la creazione grafica di modelli di simulazione, costruendo un \emph{metamodello} astratto, la cui elaborazione da parte della piattaforma genera automaticamente codice e documentazione. Oltre all'aspetto di modellazione, � offerto il supporto alla esecuzione e visualizzazione delle simulazioni basate su agenti, grazie ad una architettura modulare ed estensibile.\\
L'obiettivo di questo lavoro di tesi � quello di descrivere ed analizzare dettagliatamente la \emph{Agent Modeling Platform}, la sua architettura e le numerose tecnologie su cui si fonda, evidenziando in particolar modo le conoscenze acquisite, necessarie alla comprensione delle caratteristiche e delle problematiche riscontrate nello studio di questo ampio ed elaborato progetto.

\end{document}
