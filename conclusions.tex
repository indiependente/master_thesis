In questo lavoro di tesi è stata proposta una versione migliorata dell'algoritmo per il calcolo della \textit{partition function} Z di un sistema di Ising ferromagnetico arbitrario in \ref{chap5}, che fosse superiore nelle prestazioni rispetto a quello presentato nel lavoro \cite{rinaldi2016approximation}.\\
I risultati ottenuti durante la fase di testing, analizzati nella sezione \ref{sec:testing}, in particolare dal confronto con la versione sviluppata in \cite{rinaldi2016approximation}, rendono possibile verificare il netto miglioramento ottenuto nel \textit{running time} dell'algoritmo.\\
Tali risultati sono stati il frutto di una attenta analisi dei Teoremi, Lemmi ed assunzioni (a volte approssimative) del lavoro di Jerrum e Sinclair \cite{jerrum1993polynomial}, spesso seguita da una rivisitazione che puntasse alla correttezza ma soprattutto all'efficienza.\\
Le performance raggiunte consentono, ora, di utilizzare il lavoro sviluppato per calcolare la funzione di partizione Z in una particolare dinamica, e quindi di computare la probabilità stazionaria in tempo un \textit{polinomiale} trattabile in casi concreti, caratteristica che fino ad ora non era possibile ottenere con gli algoritmi precedenti, i quali sebbene avessero complessità polinomiale non consentivano di trattare casi reali, a causa del \textit{running time} molto sensibile anche ad un piccolo incremento nella taglia dell'input.\\
L'efficienza dell'algoritmo proposto deriva dal lavoro svolto nell'ottimizzazione dei valori \textit{s} e \textit{t} che rappresentano, rispettivamente, la taglia dell'insieme di configurazioni su cui eseguire il generatore del Teorema \ref{thm:gen} ed il numero di ripetizioni dell'esperimento necessarie su cui calcolare la mediana. Il generatore del \textit{subgraphs-world process}, descritto in \ref{sec:pf}, è stato migliorato notevolmente sia dal punto di vista dell'ottimizzazione del codice, come del resto tutto il codice implementato, ma soprattutto nella sua complessità computazionale: il precedente tempo d'esecuzione era limitato da $O(m^2\mu^{-8}(log\,\delta^{-1} + m))$, pertanto era fortemente influenzato dal numero di archi del grafo di input; grazie all'integrazione del lavoro \cite{auletta2011convergence}, come mostrato in \ref{sec:mypf}, la complessità è stata portata a $O(2m^2\mu^{-4}(log\,\delta^{-1} + 1))$, riducendo così di molto la dipendenza dal numero di archi.\\
Gli sviluppi futuri più interessanti riguardano il generatore. La direzione da prendere sarebbe quella di lavorare sulla catena di Markov da simulare ed in particolare un importante passo avanti prevederebbe la sua parallelizzazione, operando sulla matrice di transizione: tale miglioramento porterebbe un notevole incremento prestazionale, consentendo di effettuare esperimenti anche su reti sociali di elevata dimensione, così da poterne computare la magnetizzazione e ad esempio poter contribuire concretamente nel campo del \textit{computational advertising}.