\section{Teoria dei Grafi}
La teoria dei grafi è una branca della matematica, nata nel 1700 con Eulero, che consente di descrivere le relazioni che intercorrono tra un insieme di oggetti.\\
Il grafo è lo strumento attraverso il quale tali relazioni possono essere espresse ed organizzate. Infatti, il grafo, consiste di oggetti chiamati \textit{nodi} e relazioni tra coppie di questi oggetti detti \textit{archi}; nodi connessi tra loro da un arco sono detti \textit{vicini} o \textit{adiacenti}.\\


\begin{figure}
	
\end{figure}


La relazione tra una coppia di nodi può essere di due tipi:\\
\begin{itemize}
	\item Simmetrica: l'arco connette i nodi con un collegamento bidirezionale ed è detto \textit{indiretto}. Un grafo costituito di soli archi indiretti è anch'esso detto indiretto.
	\item Asimmetrica: l'arco connette i nodi con un collegamento unidirezionale ed è detto \textit{diretto}. Un grafo costituito di soli archi diretti è anch'esso detto diretto.
\end{itemize}
\begin{figure}
	\vspace*{1cm}
	\begin{minipage}{0.45\textwidth}
	\centering
	\begin{tikzpicture}[-,>=stealth',shorten >=1pt,auto,node distance=2cm,
		thick,main node/.style={circle,draw,font=\sffamily\Large\bfseries}]
		\node[main node] (1) {1};
		\node[main node] (2) [below left of=1] {2};
		\node[main node] (3) [below right of=2] {3};
		\node[main node] (4) [below right of=1] {4};
		\path[every node/.style={font=\sffamily\small}]
		(1) edge node [left] {} (4)
		
		(2) edge node [right] {} (1)
		
		(3) edge node [right] {} (2)
		
		(4) edge node [left] {} (3);
		
	\end{tikzpicture}
	\caption{Grafo indiretto}
	\end{minipage}\hfill
% <-- needed to keep the imgs side by side
	\begin{minipage}{0.45\textwidth}
	\centering
	\begin{tikzpicture}[->,>=stealth',shorten >=1pt,auto,node distance=2cm,
		thick,main node/.style={circle,draw,font=\sffamily\Large\bfseries}]
		\node[main node] (1) {1};
		\node[main node] (2) [below left of=1] {2};
		\node[main node] (3) [below right of=2] {3};
		\node[main node] (4) [below right of=1] {4};
		\path[every node/.style={font=\sffamily\small}]
		(1) edge node [left] {} (4)
		
		(2) edge node [right] {} (1)
		
		(3) edge node [right] {} (2)
		
		(4) edge node [left] {} (3);
	\end{tikzpicture}
	\caption{Grafo diretto}
	\end{minipage}
\end{figure}
\subsection{Complex Networks}
%wiki seems good


\section{Modello di Ising}
\subsection{Partition Function}

\section{Cenni di probabilità e statistica}

\section{Processi Markoviani}
\subsection{Irriducibilità e periodicità}
\subsection{Distribuzione stazionaria}
\subsection{Catena di Markov Monte Carlo}

\section{Algoritmi di approssimazione}